%%%%%%%%%%%%%%%%%%%%%%%%%%%%%%%%%%%%%%%%%
% Medium Length Professional CV
% LaTeX Template
% Version 2.0 (8/5/13)
%
% This template has been downloaded from:
% http://www.LaTeXTemplates.com
%
% Original author:
% Trey Hunner (http://www.treyhunner.com/)
%
% Important note:
% This template requires the resume.cls file to be in the same directory as the
% .tex file. The resume.cls file provides the resume style used for structuring the
% document.
%
%%%%%%%%%%%%%%%%%%%%%%%%%%%%%%%%%%%%%%%%%

%----------------------------------------------------------------------------------------
%   PACKAGES AND OTHER DOCUMENT CONFIGURATIONS
%----------------------------------------------------------------------------------------

\documentclass{resume} % Use the custom resume.cls style
\usepackage[margin=0.75in]{geometry}
%url support
\usepackage{hyperref}
% Document margins
\usepackage{amssymb}

\usepackage{fancyhdr}
\usepackage{lastpage}

\pagestyle{fancy}
\fancyhf{}

\rfoot{Page \thepage \hspace{1pt} of \pageref{LastPage}}

%----------------------------------------------------------------------------------------
%   DOCUMENT START
%----------------------------------------------------------------------------------------
\title{StengelResume}
\name{Karen ``Ren" Stengel} % Your name
\address{github: @karenlstengel \\ linkedin: @karenlstengel} % Your phone number and email
\address{karenlstengel@gmail.com \\ karen.stengel@colorado.edu}
\begin{document}
%----------------------------------------------------------------------------------------
%   EDUCATION SECTION
%----------------------------------------------------------------------------------------
\vspace{-.3cm}
\begin{rSection}{Education}
%graduate school here
\begin{rSubsection}{University of Colorado, Boulder}{\textit{Boulder, CO}}{Computer Science PhD student} \hfill \normalfont{GPA: 3.870} \newline \hspace*{\fill} August 2020 $\to$ Present %March 2019 $\to$ Present
%\vspace{-.1cm}
\vspace{-.65cm}
\item[]
\end{rSubsection}
\vspace{0.6cm}

\begin{rSubsection}{Montana State University}{\textit{Bozeman, MT}}{B.S. in Computer Science (Highest distinction)\hfill \normalfont{GPA: 3.74} \textit{ \\  B.S. in Cell Biology \& Neuroscience  (Highest distinction)\hfill \normalfont{December 2018}\\ \textit{Honors Program - Distinction}} }{}
%\vspace{-.1cm}
\vspace{-.65cm}
\item[]
\end{rSubsection}
\vspace{-.6cm}
\end{rSection}

%----------------------------------------------------------------------------------------
%   WORK EXPERIENCE SECTION
%----------------------------------------------------------------------------------------

\begin{rSection}{Work \& Research Experience}
% TODO - update CU section with PSAAP stuff, etc
\begin{rSubsection}{University of Colorado, Boulder}{August 2020 $\to$ Present}{Computer Science PhD Student}{Boulder, CO}\\
Advisor: Dr. Jed Brown\\
\textit{Solid Mechanics in libCEED and Ratel (PSAAP funded)}
\begin{itemize}
    \renewcommand\labelitemi{\textbullet}
    \setlength \itemsep{-0.4em}
    \item Implemented the Mooney-Rivlin inelastic model, quasi-static solver capabilities, and multiple materials support. Assisted with contact implementation, and plasticity models.
    \item Aided conversion of the libCEED solid mechanics mini-app to a standalone library, Ratel, and miscellaneous implementation improvements
    \item Presented work at PSAAP funding reviews - posters and short talks.
\end{itemize}
\end{rSubsection}

% TODO - update with project info
\begin{rSubsection}{Sandia National Laboratory}{May 2022 $\to$ August 2022}{Scientific Machine Learning Graduate Intern (PSAAP)}{Albuquerque, NM}\\
Advisor: Dr. Warren L. Davis\\
\textit{Predicting time dependent properties of materials}
\begin{itemize}
    \renewcommand\labelitemi{\textbullet}
    \setlength \itemsep{-0.4em}
	\item Performed main data analysis for project.
    \item Designed and built a data processing package to properly format and edit data from raw output files. Included data augmentation and I/O to TFRecords for improved data I/O during training.
	\item Designed and built a TensorFlow based package to perform learning on the provided material and output data. Designed to be flexible for architecture and parameter tuning.
\end{itemize}
Advisor: Dr. Joseph Hart\\
\textit{CLDERA - Autoencoders for dimension reduction}
\begin{itemize}
    \renewcommand\labelitemi{\textbullet}
    \setlength \itemsep{-0.4em}
    \item Expanded the functionality of the existing dimension reduction package (in CLDERA) to offer autoencoders as an alternative method to PCA.
	\item Preserved existing abstraction for software design. This provids a flexible framework to allow the user to select at runtime which autoencoder or PCA method to use.
	\item Presented work at the CLDERA group meeting as a summer wrap up.
\end{itemize}
\end{rSubsection}

\begin{rSubsection}{National Renewable Energy Laboratory}{March 2019 $\to$ August 2020}{Computational Science Center Intern}{Golden, CO}\\
Advisor: Dr. Ryan King \\
\textit{Physics-Informed Resolution Enhancing GANs}
\begin{itemize}
    \renewcommand\labelitemi{\textbullet}
    \setlength \itemsep{-0.4em}
    \item Implemented a Physics-Informed Deep Learning approach to spatially super-resolve wind and solar data from NCAR's CCSM climate model in order to increase predictive accuracy of wind and solar resources under various climate change change scenarios. Open source software: PhIREGANs.
    \item Extended the spatial super-resolution network to perform temporal Super-Resolution while maintaining accurate fluid physics and ensuring generated time steps are temporally coherent
    \item Extended Super-resolution network to generate a distribution of super-resolved images rather than one image/input
    \item Gained experience operating in an HPC environment. All Networks were run on NREL's supercomputer Eagle
    \item Improved Python-based software engineering abilities and was in charge of experimental design for both WIND Toolkit and NSRDB experiments
    \item Increased understanding of Fluid Dynamics (notably atmospheric), Computational Fluid Dynamics, vector calculus, and corresponding analysis methods
    \item Presented work at the American Geophysical Union (AGU) as well as research groups internal and external to NREL
    \item Became proficient in handling large quantities of data from several different data sets (WIND Toolkit, National Solar Radiation Database, Community Climate System Model)
    \item Worked on side projects with Dr. Jennifer King (Hybrid energy plan optimization) and Dr. Ariel Miara (Food, energy, and water (FEW) systems modeling)
\end{itemize}

\end{rSubsection}

\begin{rSubsection}{National Center for Atmospheric Research}{May 2018 $\to$ August 2018}{Summer Internship in Parallel Computational Sciences Intern}{Boulder, CO}\\
Advisor: Dr. Davide del Vento
\begin{itemize}
    \renewcommand\labelitemi{\textbullet}
    \setlength \itemsep{-0.4em}
    \item Implemented a Deep Learning approach to predict abnormally hot days over long-term time scales (20-50 days) utilizing HPC techniques on the NVIDIA K80 GPUs on the Cheyenne Supercomputer
    \item Increased proficiency in using sbatch to execute Neural Network models on the NCAR and XSEDE supercomputers
    \item Presented project as both a final talk and a poster session
    \item Participated in one day leadership training with 5.12 Solutions Consulting Group and attended weekly professional development workshops for interns
\end{itemize}
\end{rSubsection}
\begin{rSubsection}{Montana State University}{May 2014 $\to$ May 2018}{Research Assistant; Dept. of Cell Biology \& Neuroscience}{Bozeman, MT}\\
Advisor: Dr. Susy Kohout
\begin{itemize}
    \setlength \itemsep{-0.4em}
    \item Designed and created many Voltage Sensing Phosphatase (VSP) variants using PCR for single point mutation, epitope tagging, and swapping VSP and/or its subsections using the Clonetech In-Fusion protocol.
    \item Electrophysiological characterization of VSP from multiple vertebrate species expressed in \textit{Xenopus laevis} oocytes using Two Electrode Voltage Clamping.
    \item Investigated VSPs' phosphatase activities in a \textit{X. laevis} expression system using fluorescence and PH-domain based FRET assays
    \item Analyzed FRET and fluorescence data using ClampFit, IgorPro, and Microsoft Excel. Analysis included fluorescence trace normalization and averaging trials.
    \item Head frog surgeon. Trained four other surgeons, organized weekly surgeries and optimized \textit{Xenopus laevis} oocyte preparation
    \item Co-presented a poster at the Biophysical Society Annual Conference 2018
    \end{itemize}
\end{rSubsection}

\begin{rSubsection}{Montana State University}{February 2017 $\to$ November 2017}{Research Assistant; Gianforte School of Computing}{Bozeman, MT}\\
Advisor: Dr. Indika Kahanda
\begin{itemize}
        \setlength \itemsep{-0.4em}
        \item Predicted residue level protein-protein interaction sites using the PAIRPred tool
        \item Acquired package management and dependency integration skills in a Linux environment
        \item Genetics and cellular biology peer mentor
        \end{itemize}
\end{rSubsection}

\end{rSection}
%----------------------------------------------------------------------------------------
%   Publications
%----------------------------------------------------------------------------------------
\iftrue
\begin{rSection}{Publications}
% TODO - add in stability paper when applicable
% TODO - update below when confirmed.
Brown, J., Barra, V., Beams, N., Ghaffari, L., Knepley, M., Moses, W., Shakeri, R.,  \textbf{Stengel, K.}, Thompson, J. L., \& Zhang, J. ``Performance Portable Solid Mechanics via Matrix-Free $p$-Multigrid." arXiv preprint arXiv:2204.01722 (2022).

Hassanaly, M., Glaws, A., \textbf{Stengel, K.}, \& King, R. N. (2021). Adversarial sampling of unknown and high-dimensional conditional distributions. Journal of Computational Physics, 110853.

\textbf{Stengel, K.}, Glaws, A., Hettinger, D., \& King, R.N. (2020) \textit{Physics-Informed Super-Resolution of Climatological Wind and Solar Data}. Proceedings of the National Academy of Sciences. Jul 2020, 201918964; DOI: 10.1073/pnas.1918964117

Rayaprolu, V., Royal, P., \textbf{Stengel, K.}, Sandoz, G., \& Kohout, S. C. (2018). \textit{Dimerization of the voltage-sensing phosphatase controls its voltage-sensing and catalytic activity}. The Journal of General Physiology, 150(5), 683-696.
\end{rSection}
\fi
%----------------------------------------------------------------------------------------
%   Posters and conference presentations
%----------------------------------------------------------------------------------------
\iftrue
\begin{rSection}{Posters and Presentations}
\setlength \itemsep{-0.4em}

% add in PSAAP related ones
\textbf{Stengel, K.}, Glaws, A., \& King, R. ``Adversarial super-resolution of climatological wind and solar data." AI Super-resolution Simulations: From Climate Science to Cosmology workshop, Carnegie Mellon University. February 23-25, 2022 [invited]

\textbf{Stengel, K.}, Glaws, A., \& King, R. ``Adversarial super-resolution of climatological wind and solar data." ``The Datasci Group", CSU. November 17, 2020 [invited]

\textbf{Stengel, K.}, Glaws, A., \& King, R. ``Physics-Informed Super Resolution of Climatological Wind and Solar Resource Data." \textbf{American Geophysical Union}; A43E: Machine Learning for Climate Modeling I. December 12, 2019

\textbf{Stengel, K.}, Glaws, A., \& King, R. ``Physics-Informed Super-Resolution of Climatological Wind Data." \textbf{Intern Research Symposium}. August 07, 2019

\textbf{Stengel, K.}, Glaws, A., \& King, R. ``Physics-Informed Super-Resolution of Climatological Wind Data." \textbf{Rocky Mountain Fluid Mechanics Research Symposium}. July 29, 2019

\textbf{Stengel, K.}, Driscol, J., Del Vento, D., Fanfarillo, A., Sobhani, N., \& Stepaniak, D. ``Machine Learning for Long-term Weather Forecasting." Final project presentation. [Poster and Presentation] August 2018

Rayaprolu, V., Royal, P., \textbf{Stengel, K.}, Sandoz, G., \& Kohout, S. C. ``Does VSP Multimerize and Does It Matter?" \textbf{Biophysical Journal}. Volume 114, Special Issue 3, 476A, February 02, 2018

Rayaprolu, V., Royal, P., \textbf{Stengel, K.}, Sandoz, G., \& Kohout, S. C. ``Does VSP Multimerize and Does It Matter?" Montana State University Undergraduate Research Celebration. April 18, 2018

\textbf{Stengel, K.}, Kohout, S. C. ``Comparison of the Voltage Sensitive Phosphatases from Vertebrate Species." Montana State University Undergraduate Research Celebration. August 04, 2016
\end{rSection}
\fi
%----------------------------------------------------------------------------------------
%   Software Records
%----------------------------------------------------------------------------------------
\iftrue
\begin{rSection}{Software Records}
\textbf{Stengel, K.}, Glaws, A., Hettinger, D., \& King, R. PhIRE GANs [Software]. National Renewable Energy Laboratory. Apache 2.0. (2019) Available at: \url{https://github.com/NREL/PhIRE} %\\\\
%\textbf{ Glaws, A.}, Stengel, K., \& King, R. S-PLN [Software]. National Renewable Energy Laboratory. Apache 2.0. (2019) Available at: \url{https://github.com/NREL/S-PLN}
\end{rSection}
\fi
%----------------------------------------------------------------------------------------
%   Awards
%----------------------------------------------------------------------------------------
\iftrue
\begin{rSection}{Awards}
\noindent National Science Foundation Graduate Research Fellowship {\it Honorable Mention} \hfill 2022\\
Rocky Mountain Fluid Mechanics Research Symposium Best Talk [Machine Learning] \hfill 2019 \\
Montana INBRE Biomedical Research Grant
\hfill Summer 2015 $\to$ Spring 2018\\
Travel Scholarship: Biophysical Society Annual Conference \hfill 2018\\
Dean's List \hfill All Semesters\\
President's List \hfill Summer 2015, 2016; Fall 2016; Spring 2018
\end{rSection}
\fi
%----------------------------------------------------------------------------------------
%   TEACHING SECTION
%----------------------------------------------------------------------------------------
\iftrue
\begin{rSection}{Teaching Experience }
\begin{rSubsection}{Montana State University}{Fall 2017, Spring 2018}{Teaching Assistant}{Bozeman, MT}
\begin{itemize}
\setlength \itemsep{-0.4em}
    \item Instructed students during labs, review sessions, and office hours. Graded papers and exams.
\\\hspace*{5mm} Web Design with Mr. Hunter Lloyd \hfill{Spring 2018 semester}
\\\hspace*{5mm} Cellular \& Molecular Biology with Dr. Christa Merzdorf \hfill{Fall 2017 semester}
\\\hspace*{5mm} Advanced Cellular \& Molecular Biology with Dr. Thomas Hughes \hfill{Fall 2017 semester}

\end{itemize}
\end{rSubsection}
\begin{rSubsection}{Montana State University}{Fall 2018}{Course Assistant}{Bozeman, MT}
\begin{itemize}
\setlength \itemsep{-0.4em}
    \item Aided the Teaching Assistant during labs for additional help with questions.
\\\hspace*{5mm} Basic Data Structures and Algorithms with Mr. Daniel DeFrance

\end{itemize}
\end{rSubsection}
\end{rSection}
\fi
%----------------------------------------------------------------------------------------
%   Projects
%----------------------------------------------------------------------------------------
\iftrue
\begin{rSection}{Projects}

\setlength \itemsep{-0.4em}
\textbf{Sandia - CLDERA} Added autoencoder functionality to an existing PCA software package for the CLDERA project. The main objective was to make switching to an autoencoder as a PCA alternative as easy as possible at runtime. All code was fully abstracted to acheive this, as well as to allow for adjusting autoencoder architectures.

\textbf{Sandia - Deep Learning} Designed and built Python packages for data processing and deep learning. The goal of the this project was to be able to predict time dependent properties of materials based on the materials initial properties to reduce numerical simulation times.

\textbf{NREL} - Super resolution of CCSM4 climate model wind and solar data. This project utilized data from NREL's WIND Toolkit, NREL's National Solar Radiation Database, and NCAR's CESM model. Our approach was based off of the SRGAN model with the goal of super resolving the climate wind data (100km resolution) to that of the WIND Toolkit (2km resolution) or of the NSRDB (4km resolution). This work also explored using perceptual losses rather than the standard pixel-wise losses commonly used for image generation problems. All Networks were run on NREL's supercomputer Eagle. %add in graphs when applicable

\textbf{NCAR} - Long-term Weather Forecasting with Deep Learning. Designed Neural Networks for predicting hot days in the Eastern United States based on Sea Surface temperature data from the NOAA OI SST V2 High Resolution Dataset. All Networks were run on the Cheyenne Supercomputer.

%final project with Jed
\textbf{CSCI5636} - Numerical Solutions of PDEs. Did a community evaluation of PyLith Geodynamics software and aimed to implement one of Pylith's example simulations in Ratel.

%final project with Owen
\textbf{APPM5370} - Computational Neuroscience. Implemented the model in [Smolen et al. 2012] in Python and added in random noise to look at the robustness of the CaMKII to Long-term Potentiation kinase cascade. The main goal of this project was to gain experience with modeling complex protein systems and their role in synaptic plasticity.

%final project with Maggie
\textbf{CSCI5253} - Datacenter Scale Computing. Created a ski resort recommender app in Kubernetes with Docker, the REST API, the Google Maps API, the WeatherUnlocked API, and HTML scrubbing. The app would automatically find relevant travel and weather/conditions informations for each ski area and then rank them according to shortest travel time and best conditions.

% final project with Nick
\textbf{CSCI5352} - Network Analysis and Modeling. Expanded SIR and SEIR models to include a time-dependent viral load (using COVID-19 as an example virus, based off of current literature) for a more realistic epidemic simulation.

\textbf{EGEN310} - Multidisciplinary Engineering. Created a GUI for steering a remote controlled vehicle along a predetermined course.

\textbf{CSCI447} - Machine Learning. Explored various ML techniques such as Neural Networks, Evolutionary/Genetic Algorithms, and Clustering Algorithms such as Ant Colony Optimization and Particle Swarm Optimization.

\textbf{CSCI446} - Artificial Intelligence. Solved common AI problems such as Maze search, Free flow, and WUMPUS world in Java.

\textbf{CSCI468} - Compilers. Created a minimal Java compiler for the 'Little' Java language using ANTLR4 to generate the corresponding parser Java files. Wrote with my team the regular expressions, grammar, symbol tables, and assembly 'Tiny' code generation algorithms.
\end{rSection}
\fi

%----------------------------------------------------------------------------------------
%   TECHNICAL STRENGTHS SECTION
%----------------------------------------------------------------------------------------
\iftrue
\begin{rSection}{Technical Skills \& Courses }
\vspace{.2cm}
\textbf{Languages:} Python, Java, C, C++, Julia, SQL, Haskell, Prolog, MATLAB, R \& RStudio, HTML, CSS, JavaScript, \LaTeX, basic UML \& OCL \vspace{.2cm} \\
\textbf{Software \& Packages:} Anaconda, ANTLR4, Atom, BASH, Docker, Gantt, GCC, GitHub, HPC, Illustrator (Adobe Creative Cloud), IntelliJ, Keras, Kubernetes, Microsoft Office, NetBeans, PAIRpred, PyCharm, RPM package manager, scripting (general), Slack, SLURM, SSH, Tensorflow, Trello, VIM, XCode\vspace{.2cm}\\
\textbf{Operating Systems:} Linux (CentOS, Fedora, LinuxMint, RedHat, Ubuntu), MacOSX, Windows\vspace{.2cm}\\
\textbf{Other:} electronics/PC construction, CPUs, GPUs, resource allocation, virtual environments, batch jobs, dynamic programming, HPC environments \vspace{.2cm}\\
\textbf{Computer Science Courses:} Advanced Algorithms, Artificial Intelligence, Bioinformatics, Computational Biology, Compilers, Computer Theory, Databases, Datacenter Scale Design, High Performance Scientific Computing, Intermediate Technical Writing, Machine Learning,  Multidisciplinary Engineering, Network Analysis and Modeling, Systems Administration\vspace{.2cm}\\
\textbf{Neuroscience Courses:} Cognitive Neuroscience, Computational Neuroscience, Issues and Methods in Cognitive Science, Molecular Genetics, Molecular Neurological Diseases, Neuroanatomy, Neuroethology

\end{rSection}
\fi
%----------------------------------------------------------------------------------------
%   REFERENCES SECTION
%----------------------------------------------------------------------------------------
\newpage

\iftrue
\begin{rSection}{References}

\textbf{Dr. Jed Brown} \hfill Assistant Professor\\
Numerical and Scientific Computing \\
Computer Science Department
The University of Colorado at Boulder \\
1111 Engineering Drive\\
ECOT 717, 430 UCB\\
Boulder, CO 80309-0430 USA\\
phone: +1~$\cdot$~303~$\cdot$~492~$\cdot$~1592 \\
email: jed.brown@colorado.edu

\textbf{Dr. Ryan King} \hfill Senior Scientist\\
Complex Systems Simulation and Optimization \\
National Renewable Energy Laboratory \\
15013 Denver West Parkway\\
Golden, CO 80401\\
phone: +1~$\cdot$~(303)~$\cdot$~275~$\cdot$~4182\\
email: rking@nrel.gov

\textbf{Dr. Susy Kohout} \hfill Associate Professor\\
Department of Microbiology and Immunology\\
Montana State University\\
P.O. Box 173400, Bozeman, MT 59717\\
phone: +1~$\cdot$~(406)~$\cdot$~994~$\cdot$~7334\\
email: kohout.sc@gmail.com

\textbf{Dr. AJ Lauer} \hfill CISL Outreach, Diversity, \& Education (CODE) Team Lead\\
CISL \hfill SIParCS Program Director \\
NCAR\\
1850 Table Mesa Dr \\
Boulder, CO 80305\\
phone: +1~$\cdot$~(303)~$\cdot$~497~$\cdot$~1288\\
email: ajlauer@ucar.edu
\end{rSection}
\fi
%----------------------------------------------------------------------------------------
%   INTERESTS
%----------------------------------------------------------------------------------------
\iffalse
\begin{rSection}{Interests}
Cycling\hspace{1cm}Skiing\hspace{1cm}Trail and Ultra running\hspace{1cm}Ballet\hspace{1cm}Climbing
\end{rSection}
\fi
%----------------------------------------------------------------------------------------

\end{document}
